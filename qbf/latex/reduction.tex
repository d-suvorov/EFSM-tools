\documentclass[a4paper]{article}

\usepackage[top=0.5in, bottom=0.8in, left=1in, right=1in]{geometry}

\title{QBF Reduction}

\begin{document}

Notation:

\begin{enumerate}
\item $k$ is the BMC bound. All paths have length $k + 1$.
\item $i, i_1, i_2$ are FSM states.
\item $v$ is a vertex of the scenario tree.
\item $a$ is an action, $A$ is an action sequence.
\end{enumerate}

Note: guard formulae $f$ are treated as sub-events.

\section{Additional exist-variables and exist-constraints}

New variables $z_{i, z, e, f}$: whether there is action $z$ somewhere on a transition from state $i$ for action $e$ and formula $f$.

Additional constrains:

\begin{enumerate}
\item $\bigwedge\limits_{i_1} \bigwedge\limits_{(e, f)} \bigvee\limits_{i_2} y_{i_1, i_2, e, f}$: optional completeness constraint (can influence LTL semantics for wasEvent).
\item $\bigwedge\limits_{i_1} \bigwedge\limits_a \bigwedge\limits_{(e, f)} \left( z_{i_1, a, e, f} \to \bigvee\limits_{i_2} y_{i_1, i_2, e, f}\right)$: if there is an action, then there is a transition (this constraint is unnecessary if completeness is enabled).
\item $\bigwedge\limits_{v} \bigvee\limits_{i} \left(x_{v, i} \land \bigwedge\limits_{(e, f, A) \in \mathrm{EdgesFrom}(v)} \left( \left(\bigwedge\limits_{a \in A} z_{i, a, e, f}\right) \land \left(\bigwedge\limits_{a \notin A} \neg z_{i, a, e, f}\right)\right)\right)$: $z$-variables correspond to scenarios. This constraint is stronger than the constraint ``each node has at least one color''.
\end{enumerate}

\section{Forall-variables}
States of Kripke structure, as well as positions $j$ of the path correspond to the transitions of the FSM. 
\begin{enumerate}
\item $\sigma_{i, j}$: $j$-th position of the path is a transition from state $i$ of the FSM.
\item $\epsilon_{e, f, z}$: $j$-th position of the path is a transition with event $e$ and formula $f$.
\item $\zeta_{a, j}$: $j$-th position of the path is a transition with action $a$ (and possibly some other actions).
\item $h_\alpha$: optional variables for subterms $\alpha$ of the LTL formula. They are generated during formula translation. Without them, the QBF size is exponential of $k$.
\end{enumerate}

Atomic predicates can be expressed as follows:
\begin{enumerate}
\item $\mathrm{wasEvent}(e)_j = \bigvee\limits_{(e, f)} \epsilon_{e, f, j}$.
\item $\mathrm{wasAction}(a)_j = \zeta_{a, j}$.
\end{enumerate}

Note: action order is not captured by these predicates.

\section{Reduction formula and forall-constraints}

$$\exists \{x_{v, i}\}, \{y_{i_1, i_2, e, f}\}, \{z_{i, a, e, f}\} \quad \forall \{\sigma_{i, j}\}, \{\epsilon_{e, f, z}\}, \{\zeta_{a, j}\}, \{h_\alpha\}$$
$$S \land \left( \neg H \lor \neg [[M]]_k \lor \neg \left( \neg L_k \land [[g]]_k^0 \lor \bigvee\limits_{l = 0}^k \left( {_l} L_k \land {_l} [[g]]_k^0\right) \right) \right)$$

\begin{enumerate}
\item $g$ is the LTL formula to verify, which is negated and converted to negation-normal form (all negations are before atomic predicates).
\item $S$ are the constrains from scenarios (with the ones from Section 1).
\item $H = \bigwedge_\alpha (h_\alpha = \alpha)$ are optional constraints which define subterm variables.  
\item $[[M]]_k = \sigma_{0, 0} \land A_1 \land A_2 \land B \land C \land D$: the path is initialized (starts from state 0 of the FSM) and is correct. Correctness constraints $A_1, A_2, B, C, D$ are defined below.
\item $L_k = \bigvee\limits_{l = 0}^k {_l} L_k$: the path is looping for some $l$.
\item ${_l} L_k = \bigvee\limits_{i_1} \bigvee\limits_{i_2} \bigvee\limits_{(e, f)} \left( \sigma_{i_1, k} \land \epsilon_{e, f, k} \land \sigma_{i_2, l} \land y_{i_1, i_2, e, f} \right)$: there exists a loop from the last position of the path to some position $l$. Note that this looping edge is not included in the path, and thus $y_{i_1, i_2, e, f}$ might not hold if removed from the constraint.
\item $A_1 = \bigwedge\limits_{j = 0}^k \left( \bigwedge\limits_{\{i_1 \ne i_2\}} \neg \left( \sigma_{i_1, j} \land \sigma_{i_2, j}\right)\right)$: there is no transition in the path from some two states simultaneously.
\item $A_2 = \bigwedge\limits_{j = 0}^k \left( \bigwedge\limits_{\{(e_1, f_1) \ne (e_2, f_2)\}} \neg \left( \epsilon_{e_1, f_1, j} \land \epsilon_{e_2, f_2, j}\right)\right)$: there is no transition in the path for some two $(e, f)$-pairs simultaneously.
\item $B = \bigwedge\limits_{j = 0}^k \left( \left(\bigvee\limits_i \sigma_{i, j}\right) 
\land \left(\bigvee\limits_{(e, f)} \epsilon_{e, f, j}\right) \right)$: each transition in the path is from some state $i$ and for some $(e, f)$-pair.
\item $C = \left( \bigwedge\limits_{j = 0}^{k - 1} \bigwedge\limits_{i_1} \bigwedge\limits_{i_2} \bigwedge\limits_{(e, f)} \left( \sigma_{i_1, j} \land \epsilon_{e, f, j} \land \sigma_{i_2, j + 1} \to y_{i_1, i_2, e, f}\right)\right)\land \left( \bigwedge\limits_{i_1} \bigwedge\limits_{(e, f)} \left( \sigma_{i_1, k} \land \epsilon_{e, f, k} \to \bigvee\limits_{i_2} y_{i_1, i_2, e, f}\right)\right)$: transitions in the path correspond to $y$-variables (the path is indeed a path in the Kripke structure). The part after the topmost AND is not required, if completeness constrain is present (Section 1).
\item $D =  \bigwedge\limits_{j = 0}^k \bigwedge\limits_{i_1} \bigwedge\limits_{a} \bigwedge\limits_{(e, f)} \left( \sigma_{i_1, j} \land \epsilon_{e, f, j} \to \left( \zeta_{a, j} = z_{i_1, a, e, f}\right)\right)$: each edge in the path must have correct (corresponding to $z$-variables) actions.
\item $[[g]]_k^0$ and ${_l}[[g]]_k^0$ are formula translations (see sections 2.4--2.5 of P. Jackson, D. Sheridan. A compact linear translation for bounded model checking. 2006).
\end{enumerate}
\end{document}
